\chapter{Sample programs}
\label{sample}
In this appendix we collect together some of the complete programs from the main text.

\section{Directory}
The directory example is developed in Chapter~\vref{directory}. It is also a key component of the ballroom example from Chapter~\vref{dance}.
\begin{alltt}
directory\{
  import go.io.
  import go.mbox.
  import go.dynamic.

  attVal ::= none.         -- Constructors given externally

  attribute ::= attr(symbol,attVal).

  private
  DSreply ::= inform(list[list[attribute]]) | ok.

  private
  DSmessage ::= reg(list[attribute],dropbox[DSreply])
    | search(list[attribute],list[symbol],
             dropbox[DSreply]).
              
  directory \typearrow \{
    find:[list[attribute],list[symbol]]=>
           list[list[attribute]].
    register:[list[attribute]]*
  \}.

  private dir_server:[mailbox[DSmessage]]@>thread.
  dir_server(_) <= thread().
  dir_server(Mbx)..\{
    start() -> directory_server().

    creator() => this.

    matches:[list[t],list[t]]\{\}.
    matches(D,E) :-
        A in E *> A in D.
    
    description:dynamic[list[attribute]]=dynamic([]).

    directory_server:[]*.
    directory_server() ->
      case Mbx.next() in (
        reg(Descr,Rep) ->
          description.add(Descr);
          Rep.post(ok)
      | search(SDescr, AttrNms,Client) ->
          Client.post(inform(\{ extract(Desc,AttrNms) ..
            (Desc::matches(Desc,SDescr)) in 
                description.ext()\}))
      );
      directory_server().

    extract:[list[attribute],list[symbol]]=>
                list[attribute].
    extract(Descr,Nms) =>
      \{attr(Nm,A) .. (Nm::attr(Nm,A) in Descr) in Nms\}.
  \}.

  private
  dBox:mailbox[DSmessage] = mailbox().

  \$\{
    -- We create the standard server
    -- when the directory package is loaded
    _ = dir_server(dBox)
  \}.

  client:[]@>directory.
  client()..\{
    mH:mailbox[DSreply] = mailbox().

    register(Desc) ->
      sync\{
        dBox.dropbox().post(reg(Desc,mH.dropbox()));
        mH.msg(ok)
      \}.
      
    find(Desc,Atts) =>
      valof\{
        sync \{ 
          dBox.dropbox().post(search(Desc,Atts,
                              mH.dropbox()));
          mH.msg(inform(L));
        \};
        valis L
     \}
  \}
\}
\end{alltt}

\section{Ballroom}
\label{sample:ballroom}
The ballroom example is developed in Chapter~\vref{dance}. This program references the package \q{directory} which is defined above.

\begin{verbatim}
/*
 * The dance card scenario
 */
 
dance{
  import go.io.
  import go.stdparse.
  import go.unit.
  import go.mbox.
  import directory.
  
  danceProto ::= shallWe(symbol,dropbox[danceProto]) | 
                 whyNot(symbol,dropbox[danceProto]) | 
                 rainCheck(symbol,dropbox[danceProto]).
  
  dancer<~ {toTheDance:[directory]*. name:[]=>symbol}.

  -- define attribute value type locally
  drop:[dropbox[danceProto]]@=attVal.
  drop(X)..{
    show()=>X.show().
  }.

  gender:[symbol]@=attVal.
  gender(X)..{
    show()=>explode(X).
  }.

  id:[string]@=attVal.
  id(X)..{
    show()=>X.
  }.

  phemail:[integer]@>dancer.
  phemail(Limit)..{
    partners:list[(symbol,dropbox[danceProto])] := [].
    name()=>implode("Female "<>Limit.show()).

    toTheDance(dir) ->
      D = mailbox();
      dir.register([attr('gender',gender('phemail')),
                    attr('loc',drop(D.dropbox())),
                    attr('id',id("Female "<>
                         Limit.show()))]);
      stdout.outLine(explode(name())<>
                     " ready to dance with "<>
                     Limit.show()<>" partners");
      dance(D).

    dance:[mailbox[danceProto]]*.
    dance(D)::listlen(partners)<Limit ->
      case D.next() in (
        shallWe(Who,Mail) ->
          partners := [(Who,Mail),..partners];
          Mail.post(whyNot(name(),D.dropbox()));
          dance(D)
      ).
    dance(D) -> 
      stdout.outLine(explode(name())<>
                     " has cards marked for "<>
                     {Who..(Who,_)in partners}.show());
      hangOut(D).

    hangOut:[mailbox[danceProto]]*.
    hangOut(D) ->
      case D.next() in (
        shallWe(_,M) ->
             M.post(rainCheck(name(),D.dropbox()))
       );
       hangOut(D).
  }.
  
  mail:[symbol]@>dancer.
  mail(Name)..{
    dancers:list[(symbol,dropbox[danceProto])] := [].
    name() => Name.
    
    toTheDance(dir) ->
      L = dir.find([attr('gender',gender('phemail'))],
                     ['loc','id']);
      Bx = mailbox();
      ( E in L *>
        markCard(E,Bx));
      stdout.outLine("Dancer "<>explode(Name)<>
                     " has marked the cards of "<> 
                     {Who..(Who,_)in dancers}.show()).

    markCard:[list[attribute],mailbox[danceProto]]*.
    markCard(Desc,Dx)::attr('loc',drop(P)) in Desc ->
      P.post(shallWe(Name,Dx.dropbox()));
      case Dx.next() in (
        whyNot(Who,P) -> 
          dancers := [(Who,P),..dancers];
          stdout.outLine(explode(Name)<>
                         " marked the card of "<>
                         explode(Who))
      | rainCheck(Who,P) -> 
          stdout.outLine(explode(Who)<>
                         " too busy to go with "<>
                         explode(Name))
      ).
    markCard(Desc,_) ->
        stdout.outLine(Desc.show()<>" has no location").
  }.

  ballRoom:[integer,integer]*.
  ballRoom(M,F) ->
      Dx = client();			 -- directory client
      spawnFems(F,Dx);
      waitForMen(spawnMen(M,Dx)).
    
  spawnFems:[integer,directory]*.
  spawnFems(0,_) -> {}.
  spawnFems(K,D) -> 
    spawn{ phemail(K).toTheDance(D)};
    spawnFems(K-1,D).
  
  spawnMen:[integer,directory]=>list[thread].
  spawnMen(0,_) => [].
  spawnMen(K,D) => 
    [spawn{
       mail(implode("Male "<>K.show())).toTheDance(D)
     },..spawnMen(K-1,D)].
  
  waitForMen:[list[thread]]*.
  waitForMen([]) -> {}.
  waitForMen([M,..en]) ->
      waitfor(M);
      stdout.outLine(M.show()<>" done");
      waitForMen(en).

  dancetest:[integer,integer]@=harness.
  dancetest(_,_) <= harness.
  dancetest(M,F)..{
    doAction() ->
        ballRoom(M,F)
  }.

  main([]) ->
      checkUnit(dancetest(4,4)).
  main([M,F]) ->
      checkUnit(dancetest(naturalOf%%M,naturalOf%%F)).

}.
\end{verbatim}

\section{Sudoku}
\label{sample:sudoku}
The sudoku program is developed in Chapter~\vref{sudoku}. It consists of two packages: the \q{square} package which implements the basic square array, and the main solver itself.

\subsection{Sudoku square}
\label{sample:sudoku:square}
The \q{square} package implements a polymorphic square array package: offering access to rows, columns and quadrants of an array. The elements of the array are represented using a linear list; this is clearly not the most efficient; however, it was sufficient for the needs of the solver.

\begin{verbatim}
square{
  -- Implement a model that can access a square
  -- as a set of rows, a set of columns 
  -- or a set of quadrants
  import go.showable.
  import go.io.
  import go.stdparse.

  square[t] <~ { 
        colOf:[integer]=>list[t]. 
        rowOf:[integer]=>list[t].
        quadOf:[integer]=>list[t].
        cellOf:[integer,integer]=>t.
        row:[integer]{}.
        col:[integer]{}.
        quad:[integer]{}.
      }.

  square:[list[t]]@>square[t].
  square(Init)..{
    Els:list[t] = Init.
    Sz:integer = itrunc(sqrt(listlen(Init))).
    Sq:integer = itrunc(sqrt(Sz)).
    S2:integer = Sq*Sq*Sq.

    intRange:[integer,integer+]{}.
    intRange(1,_).
    intRange(J,Mx) :- Mx>1,intRange(K,Mx-1),J=K+1.

    row(I) :- intRange(I,Sz).
    col(I) :- intRange(I,Sz).
    quad(I) :- intRange(I,Sz).

    rowOf(I) => front(drop(Els,(I-1)*Sz),Sz).

    colOf(Col) => clOf(drop(Els,Col-1)).

    clOf:[list[t]] => list[t].
    clOf([]) => [].
    clOf([El,..L]) => [El,..clOf(drop(L,Sz-1))].

    -- The expression
    -- ((i-1) quot sqrt(Sz))*Sz^1.5+((i-1)mod sqrt(Sz))
    -- is the index of the first elemnt of the ith quadrant
    -- Sz^1.5 is the number of elements in a quadrant
    quadOf(i) => valof{
                 Ix = ((i-1) quot Sq)*S2+imod((i-1),Sq)*Sq;
                 valis qdOf(drop(Els,Ix),Sq)
               }.

    qdOf:[list[t],integer] => list[t].
    qdOf(_,0)=>[].
    qdOf(Table,Cnt) => 
        front(Table,Sq)<>qdOf(drop(Table,Sz),Cnt-1).
    
    cellOf(Row,Col) => nth(Els,(Row-1)*Sz+(Col-1)).

    show() => showSquare(Els).flatten("").

    showSquare:[list[t]]=>dispTree.
    showSquare(L) => n(showRow(0,L)).

    showRow:[integer,list[t]]=>list[dispTree].
    showRow(Sz,[]) => [s("|\n")].
    showRow(Sz,L) => [s("|\n"),..showRow(0,L)].
    showRow(Ix,[El,..L]) => 
      [s(El.show()),..showRow(Ix+1,L)].
  }.

  parseCell[t] <~ { parse:[t]-->string }.

  parseSquare:[square[t],parseCell[t]]-->string.
  parseSquare(Sq,P) --> 
        skip(), parseList(L,P), Sq = square(L).

  parseList:[list[t],parseCell[t]]-->string.
  parseList([],_) --> "[]".
  parseList([El,..Rest],P) -->
        "[", skip(), P.parse(El), skip(),
             parseMore(Rest,P).

  parseMore:[list[t],parseCell[t]]-->string.
  parseMore([El,..Rest],P) --> 
        ",", P.parse(El), skip(), parseMore(Rest,P).
  parseMore([],_) --> "]".
  parseMore([],_) --> eof.

  skip:[]-->string.
  skip() --> [X],{whiteSpace(X)}, skip().
  skip() --> "#",skipToEol(),skip().
  skip() --> "".

  skipToEol:[]-->string.
  skipToEol() --> "\n".
  skipToEol() --> [C],{__isZlChar(C)}.
  skipToEol() --> [C], {\+__isZlChar(C)}, skipToEol().
  skipToEol() --> eof.

}
\end{verbatim}

\subsection{Sudoku solver}
\label{sample:sudoku:solver}
The solver included here is an extension of the solver developed in Chapter~\vref{sudoku}. Mainly it includes code that can be used to parse a string representation of the problem and it also includes code for displaying results.

\begin{verbatim}
/* A program to solve sudoku puzzles
   (c) 2006 F.G. McCabe
 
   This program is free software; you can redistribute it 
   and/or modify it under the terms of the GNU General 
   Public License as published by the Free Software 
   Foundation; either version 2 of the License, or
  (at your option) any later version.

  This program is distributed in the hope that it will be 
  useful, but WITHOUT ANY WARRANTY; without even 
  the implied warranty of MERCHANTABILITY or FITNESS 
  FOR A PARTICULAR PURPOSE.  See the GNU General 
  Public License for more details.

  You should have received a copy of the GNU General 
  Public License along with this program; if not, write to 
  the Free Software Foundation, Inc., 59 Temple Place, 
  Suite 330, Boston, MA  02111-1307  USA

  Contact: Francis McCabe <frankmccabe@mac.com>
 */
sudoku{
  import go.io.
  import go.setlib.
  import go.showable.
  import go.unit.
  import square.
  import go.stdparse.

  -- Each cell in the table is represented by a constraint
  
  constr <~ { curr:[]=>list[integer]. 
      remove:[integer]*. 
      assign:[integer]*. 
  }.
  
  constr:[list[integer]]@>constr.
  constr(L0)..{
    L:list[integer] := L0.

    curr()=>L.
    remove(K) ->
        ( K in L ? 
            L := L \ [K]).
    assign(K) ->
        ( K in L ? 
            L := [K]).

    show()::L=[] => " {} ".
    show() => " "<>L.head().show()<>sh(L.tail(),L.head()).

    sh:[list[integer],integer]=>string.
    sh([],_) => "".
    sh([I,..Ir],K)::I=K+1 => "-"<>cSh(Ir,I).
    sh([I,..Ir],_) => ","<>I.show()<>sh(Ir,I).
    
    cSh:[list[integer],integer]=>string.
    cSh([],I) => I.show().
    cSh([I,..Ir],J)::J+1=I => cSh(Ir,I).
    cSh(Ir,I) => I.show()<>sh(Ir,I).
  }.

  let[t] <~ { val:[]{}. }.

  cycle:[square[constr]]{}.
  cycle(T):-
      (:let[logical]..{
         done :logical := true.

         filter:[list[constr]]*.
         filter(Set) ->
             ( El in Set *>
               ( El.curr()=[K] ?
                 ((Ot::Ot.curr()!=[K]) in Set, 
                         K in Ot.curr() *>
                    Ot.remove(K);
                    done := false
                 )
               )
             );
             ( unique(Set,Ot,U) *>
               Ot.assign(U);
               done:=false
             ).

         filterRows:[square[constr]]*.
         filterRows(Table) ->
             ( Table.row(i) *>
               filter(Table.rowOf(i))).

         filterCols:[square[constr]]*.
         filterCols(Table) ->
             ( Table.col(i) *>
               filter(Table.colOf(i))).
         
         filterQuads:[square[constr]]*.
         filterQuads(Table) ->
             ( Table.col(i) *>
               filter(Table.quadOf(i))).

         val() :-
             action{
               filterRows(T);
               filterCols(T);
               filterQuads(T);
               istrue done
             }.
       }).val().

  solve:[square[constr]]*.
  solve(Table)->
      ( cycle(Table) ?
          {}
      | stdout.outLine("After cycle");
        stdout.outLine(Table.show());
        solve(Table)).


  -- unique of a list of lists of integers L
  -- is true of some U 
  -- if U is in only one of the non-trivial sets of L
  -- A set is non-trivial if it has more than one element
  unique:[list[constr],constr,integer]{}.
  unique(L,O,U) :- 
      append(F,[O,..B],L),
      Cn = O.curr(),
      listlen(Cn)>1,
      U in Cn, \+ (E in F, U in E.curr()),
      \+ (E in B, U in E.curr()).

  parseConstr:[list[integer]]@>parseCell[constr].
  parseConstr(Deflt)..{
    parse(C) --> 
        naturalOf(I), 
        {I = 0 ? C=constr(Deflt) | C = constr([I])}.
  }.

  parseSudoku:[square[constr]]-->string.
  parseSudoku(Sq) --> skip(), naturalOf(I), skip(),",",
      skip(), parseSquare(Sq,parseConstr(iota(1,I))).

  testSudoku:[string]@=harness.
  testSudoku(_)<=harness.
  testSudoku(Text)..{
    doAction() ->
        T = parseSudoku%%Text;
        stdout.outLine("Original");
        stdout.outLine(T.show());
        solve(T);
        stdout.outLine("Solved");
        stdout.outLine(T.show()).
  }.

  main(_) ->
      checkUnit(testSudoku("9,[0,0,1,0,2,0,0,0,0,"
                           "0,0,7,5,0,0,0,0,0,"
                           "0,3,0,0,0,0,1,7,0,"
                           "0,0,0,0,0,0,5,0,0,"
                           "1,0,6,0,0,0,9,0,2,"
                           "0,0,5,9,3,8,6,0,0,"
                           "0,0,0,0,0,0,0,8,4,"
                           "3,0,4,0,1,0,2,9,0,"
                           "0,0,0,4,0,0,3,0,1]")).
  main(_) ->
      checkUnit(testSudoku("9,[0,0,0,0,0,2,1,7,0,
                           0,0,7,0,0,5,0,0,0,
                           0,0,0,0,6,0,0,0,0,
                           0,0,4,9,0,0,0,0,0,
                           0,2,0,1,0,0,0,9,8,
                           0,0,0,0,5,4,0,0,6,
                           0,8,0,5,3,0,0,6,0,
                           0,7,3,2,0,9,0,0,0,
                           0,1,5,0,0,0,3,0,9]")).
}
\end{verbatim}