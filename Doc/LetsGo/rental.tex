\chapter{Renting cars}
\label{rental}

In a world of services and integration across the Internet a primary purpose of the technology is to enable people to carry out \emph{tasks}; especially tasks that involve distributed resources. One of the many tasks that we have to go through when planning a trip is arranging local transportation. If our destination is a major city, like London or New York, then we can rely on the local infrastructure -- such as subways, buses and taxis. However, in many places and for many purposes a car rental is often the best choice.

Planning a car rental may involve a significant amount of effort -- comparing prices and features and availability across multiple car rental agencies is tedious and time consuming. Having tools to simplify that process could be a great help to travelers.

A car rental is a classic example of a perishable good --  a car that has not been rented on a given day represents an irretrievable loss: the car-rental-day cannot be recovered. Hence, a car rental service has a desire to maximize the utilization of its fleet, whilst still providing sufficient service so that its customers continue to be loyal.

\section{The Car Rental scenario}
\label{rental:scenario}

Car rental services, like many others, often attempt to boost their revenue by providing a higher-quality of service to their customers. Fundamentally, that means meeting more of the customers' needs than the basic means of transportation. Classic examples of this include in-car navigation systems, mobile telephones, alternate drop-off locations, child safety seats and chauffeurs; none of which may be `essential' to the core business, but any of which may significantly increase the utility of the car to the customer.

Additional features may also involve the car rental agency attempting to predict the \emph{use} to which the customer is putting the car to, and supporting that -- for example with ski packages, or even boat trailers. The range of possible extensions to the core car rental is very large; and essentially open: a car rental service has every incentive to be opportunistic, responding to the dynamic business conditions a given rental office is in. 

Of course, this opportunism applies to customers as well as car rental agencies. A passenger stepping off a plane may decide that the weather is good enough to warrant hiring a car to go to the beach!

\subsection{A conversation with a car rental agency}
Imagine that we had an agent that was an expert in hiring cars. And suppose that we were going skiing in a part of the world that we had never been to before, and needed a car. If our agents were really able to work for us, we might be witness to a conversation such as:
\begin{alltt}
ourAgent: I need a car
  RentalAgent: When, and where?
ourAgent: Denver, CO, March 10th
  RentalAgent: Which class of car?
ourAgent: Big and Heavy
  RentalAgent: Thats fine; perhaps an SUV?
ourAgent: How much?
  RentalAgent: Not too expensive. 
  RentalAgent: Maybe you'd like "Map Me"\tm?
ourAgent: How did you know?
  RentalAgent: Its on special also.
ourAgent: What's the total?
  RentalAgent: \$xxx
ourAgent: I'll take it.
  RentalAgent: There's a 10\% discount if you book today
ourAgent: I'll take that too.
\end{alltt}
Of course, a conversation between computer systems if likely to be more like that in Figure~\vref{rental:conversation}.
\begin{figure}
\begin{alltt}
ourAgent: inform(newTopic("SRO::carRental"))\note{Standard Rental Ontology}
  RentalAgent: request(inform(valueOf("SRO::location")))
OurAgent: inform(equals(valueOf("SRO::location"),
                        "SGO::Denver,CA"))\note{Standard Geographic Ontology}
  RentalAgent: request(inform(valueOf("SRO::rentalStartDate")))
ourAgent: inform(equals(valueOf("SRO::rentalStartDate"),
                        "SDO::3-10-03"))
  RentalAgent: request(inform(valueOf("SRO::rentalEndDate")))
ourAgent: inform(equals(valueOf("SRO::rentalEndDate"),
                        "SDO::3-15-03"))
  RentalAgent: request(inform(valueOf("SRO::carClass")))
ourAgent: inform(if(conj([classOf(Car,X),sizeOf(X,Sz),
                          isBigNheavy(Sz)]),
                   equals(valueOf("SRO::carClass"),Car)))
  RentalAgent: inform(conj([classOf("SUV","SRO::SUV-Class"),
                          sizeOf("SRO::SUV-Class",Sz),
                          isBigNheavy(Sz)]))
ourAgent: request(inform(valueOf("SRO::rentalPrice")))
  RentalAgent: inform(equals(valueOf("SRO::rentalPrice"),
                             dollar(1000)))
  RentalAgent: inform(available("AutoRent::MapMe"))
ourAgent: request(valueOf("SRO::inCarNavigationSystem"))
  RentalAgent: inform(equals(valueOf("AutoRent::MapMe")
                             dollar(56)))
\ldots  
\end{alltt}
\caption{A conversation between agents\label{rental:conversation}}
\end{figure}

\subsection{Elements of negotiation}
One of the differences between a conversation such as that above and most conventional inter-machine conversations is that it is much more open. Although there is a significant amount of request-response style traffic, there is also an element of \emph{information volunteering}.

