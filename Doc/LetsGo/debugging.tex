\chapter{Debugging \go programs}
\label{debugging}
\index{debugging}
\lettrine[nindent=0.1em]{I}{n order} to debug a \go program\note{Debugging \go
  programs is not in its final form, as of this writing.} it is necessary to
compile it with debugging options enabled. The simplest option is the \q{-g}
flag. This results in the compiler adding additional code to the program to
generate debugging information for all programs in the package. It also results
in a substantial expansion of the code size; and a concomitant reduction in
performance.

Often it is not necessary to debug all the programs in a package. The \q{-gb \emph{PrgName}} option turns on debugging for just \q{\emph{PrgName}}.\note{The compiler is a little crude about this: it will turn on debugging for \emph{any} program called \q{\emph{PrgName}}.}

The \q{-gb} option can be used any number of times, to enable different functions to be debugged. In fact, the \q{-g} option is short for \q{-gb *}.

When a program is executed, by default no debugging information is generated -- even if it is compiled with debugging code enabled. To actually generate debugging traces and actions you must execute the \go program using the \q{-g} flag.


\section{Using the default \q{go.debug} package}
\label{debug:default}
\index{using the built-in debugger}

The default debugger is a simple debugger that allows source-level tracing of the program being debugged and for simple displays of the internal state of the system. 

The default debugger will not generate any debugging trace unless the \go program is invoked with the \q{-g} option:
\begin{alltt}
\emph{UnixPrompt}\% go -g \emph{package}
\end{alltt}

Once invoked, it displays information in the form:
\begin{alltt}
[\emph{thread-name}] \emph{flag} \emph{name} \emph{args}
\end{alltt}
where \emph{thread-name} is a short form of the thread \q{handle} currently executing, \emph{flag} is a short string indicating the kind of debugging message:
\begin{description}
\item[prove] This is used when calling a predicate. The form of the message displayed is analogous to:
\begin{alltt}
[thread#0] prove pred(wall_front(_839BFC))
\end{alltt}
The arguments to the call show their values just before entering the predicate.

\item[proved] This is used when a predicate succeeds. The form of the message displayed is analogous to:
\begin{alltt}
[thread#0] proved pred(wall_front(false))
\end{alltt}
The arguments to the call show their values just after successfully solving the predicate call.

\item[retry] This is used when a predicate or grammar call has to be re-tried. This normally aries if the call was successfully solved at least once and a subsequent failure forces a re-try -- to find an alternative solution. The form of the message displayed is analogous to:
\begin{alltt}
[thread#0] retry pred(wall_front(false))
\end{alltt}
The arguments to the call show their values just before attempting another solution of the predicate call.

\item[not proved] This is used when a predicate fails. This arises if either the predicate fails to find any solution, or if a subsequent attempt to find an alternative solution fails.  The form of the message displayed is analogous to:
\begin{alltt}
[thread#0] not proved pred(wall_front(_839BFC))
\end{alltt}
The arguments to the call show their values just before failing the predicate call -- which is also the same as their values before entering the goal.

\item[eval] This is used when calling a function.  An \q{eval} debug message is analogous to:
\begin{alltt}
[thread#1] eval goldCell.get ()
\end{alltt}
The arguments to the call show their values just before calling the function.

\item[return] This is used when reporting the result of a function call. The form of a \q{return} message is:
\begin{alltt}
[thread#1] return goldCell.get ((2, 3))
\end{alltt}
Note that the \q{return} message displays the \emph{result} of the function, not the arguments of the call.

\item[exec] This is used when entering an action procedure. An \q{exec} debug message is analogous to:
\begin{alltt}
[thread#1] exec stdout.outLine ("Gold in cell: (2, 3)")
\end{alltt}

\item[finish] This is used when an action procedure has completed. A \q{finish} debug message is analogous to:
\begin{alltt}
[thread#1] exit stdout.outLine ("Gold in cell: (2, 3)")
\end{alltt}
This will show the state of any arguments after the execution of the action procedure.

\item[parse] This is used when entering a grammar non-terminal.
\item[parsed] This is used when reporting a successful parse.
\item[defn] This is used when reporting the value of a variable.
\item[line] This is used when reporting that control is at a line in the source program. \go also displays the \emph{line} information so indicate the current source file and line number representing the location of the program.
\end{description}

When the debugger pauses for input, a range of commands may be accepted:
\begin{description}
\item[n \emph{count}]
This moves the program for \emph{count} lines before pausing again.
\item[f]
This causes the debugger to focus on the current thread and to ignore any debugging information in other threads.
\item[u]
This causes the debugger to un-focus, and to display debugging information for all threads that are executing programs with debugging code compiled.
\item[q]
This causes the debugger to force the entire \go session to terminate.
\item[t]
This causes the debugger to switch to a tracing mode -- debugging information is displayed but the user is never prompted again.
\item[c]
This causes the debugger to stop displaying debugging information. Program execution continues as though debugging were disabled. Note that there is still some performance degradation since the run-time system is still executing `debugging code'.
\item[a \emph{reg}]
Shows the contents of an argument register.
\item[y \emph{reg}]
Shows the contents of a local variable register.
\item[e \emph{off}]
Shows the contents of a free variable.
\item[r]
Shows the contents of all the available registers.
\item[m]
Shows any messages that are pending for the current thread.
\item[p]
Displays information about all the active threads in the system.
\item[P]
Displays information about all the threads in the system; including inactive threads.
\item[s]
Displays a `stack trace' of the current process.
\item[?]
Any other input will result in a reminder of the available commands being displayed.
\end{description}

\section{Using \go with Emacs}
\label{emacs:go}

There is some support for using Emacs to edit and debug \go programs. The \q{go} mode understands the \go syntax sufficiently to make formatting programs simpler and included with the installation is a simple interface to the \go debugger that allows you to watch the execution of a program in the editor window.

\subsection{Setting up for emacs}
\label{emacs:setup}



\subsection{\q{go-mode}}
\label{emacs:go-mode}
