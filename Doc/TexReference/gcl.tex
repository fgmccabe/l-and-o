\chapter{The \gcl}
\label{gcl}

The \gcl is a language that is intended for use between \go systems; particularly agent systems. It is not a programming language in the normal sense of the word -- but a language that is designed to facilitate communication of semantically grounded content.\footnote{This chapter documents an experimental extension to \go.}

\gcl has a number of distinct layers within it:
\begin{description}
\item[the propositional layer] has constructs that correspond to propositional forms, such as truth-valued formulae and action expressions,
\item[the communicative act layer] encapsulates the `speech act' that represents the basic units of a conversation,
\item[the contract layer] encapsulates the notion of an agreement. An agreement is a mutually binding constraint on the behavior of two or more entities, and\footnote{A binding agreement on a single entity is somewhat moot by definition.}
\item[the entity layer] encapsulates the notion of a free entity. Entities include individual agents, resources mediated by software and groups of entities.
\end{description}

\gcl is strongly typed, as befits a \go extension. However, it is in a sense merely a shell into which application-specific information must be embedded. The `root' types for the \gcl are \q{propForm} which defines the legal forms of logical proposition in \gcl, \q{ontForm} which defines the legal form of an ontology reference, \q{caForm} which defines the legal communicative acts, \q{actForm} which defines the legal action expressions, and ctForm which defines the legal form of an agreement or contract. Each of these types are polymorphic, and are expected to be `instantiated' with application-specific -- i.e., your -- types that represent the domain information.



For example, if \q{agent1} wishes to inform \q{agentX} that the color of a particular car is blue, then it will send a message of the form:
\begin{alltt}
inform(\emph{agentX},
   ?+(color(?!(X,&(?+(topicOf('abc\#234154',X)),
                  ?+(isA(X,
                      inOnt('org.gVM','car'))))),
            inOnt('org.intColor','blue')))
)
\end{alltt}
where \q{inform} means `I assert the truth of', \q{?+} means `proposition', \q{?!(X,$\phi$)} means `the \q{X} that satisfies $\phi$, \q{topicOf} is an application predicate that refers to the  topic of some ongoing conversation, \q{inOnt} is a constructor that links a symbol to a particular ontology.

Informally, we can read such a message as:
\begin{quote}
I assert that the color of the entity that is the topic of conversation \q{abc\#234154}, which is also a car -- as defined by the \q{gVM} org (global Vehicle Manufacturer's association) -- is \q{blue} -- in turn as defined by the international Color organization.\footnote{Any resemblance to actual organizations is purely coincidental.}
\end{quote}

\section{The propositional forms}
\label{gcl:propositional}

The propositional forms in are used to denote logical formulae. \gcl is a rule-based language, rather than admit arbitrary shaped propositional forms, \gcl propositions all take the form of rules. This does not limit the logical expressiveness of the language; however, it does make interpretation significantly simpler.

The legal formulae are those of the \q{propForm} type:
\begin{alltt}
-- An agent's name has no structure
agent ::= aId(symbol).


-- CT is a content ontology, AT an action ontology and ST a social ontology
propForm(CT,AT,ST) ::=  TRUE |
		?+(CT)  | ?\nasf(CT) | ?=(T,T) | ?\bsl{}=(T,T) |
       or(propForm(CT,AT,ST),propForm(CT,AT,ST)) | 
		&(propForm(CT,AT,ST),propForm(CT,AT,ST)) | 
		not(propForm(CT,AT,ST)) | 
	    md(modal(CT,AT,ST)).

-- A rule is either a logical rule or a modal rule
rule(CT,AT,ST) ::= rl(CT,propForm(CT,AT,ST)) |
       cl(modal(CT,AT,ST),propForm(CT,AT,ST)).

-- modal conditions are somewhat different from regular predicates
  modal(CT,AT,ST) ::= 
    allowed(ST,agent,AT) | obliged(ST,agent,AT) |    -- deontic operators
    feasable(agent,AT) | done(agent,AT,propForm(CT,AT,ST)) |       -- action operators
    authorized(ST,agent,propForm(CT,AT,ST)) |        -- empowerment operators
    entitled(ST,agent,propForm(CT,AT,ST)).
\end{alltt}

\subsection{\q{aId} -- agent identifier}
\label{gcl:aid}

An agent is identified by a simple structure which merely encapsulates a symbol. For the purposes of the \gcl, agent names have no internal structure -- their main r\^ole is to allow reasoning about agents across sequences of messages.

\subsection{\q{propForm} -- the proposition type}

\subsection{\q{?+} -- leaf proposition}
\label{gcl:leaf}

The \q{?+} constructor denotes a leaf proposition; corresponding to base predicates in the domain. The form of the \q{?+} constructor is:
\funsynopsis{?+(\param{A}:\type{CT})}{\type{propForm(CT,AT,ST)}}
where \param{A} is the atomic proposition and \type{CT} is the type of such propositions. Typically, \type{CT} will be a type in the user's domain; for example, the expression:
\begin{alltt}
?+(male('f'))
\end{alltt}
would be of type \q{propForm(people,T\sub1,T\sub2)} if \q{male} is of type \q{people}:
\begin{alltt}
people ::= male(symbol) | \ldots
\end{alltt}
